%! suppress = LineBreak
\specialsection{Введение}

С тех пор как были изобретены компьютерные игры, количество человеко-месяцев, которые уходят на разработку успешной коммерческой игры, постоянно увеличивалось. Сейчас является нормой, что игра разрабатывается сотнями людей на протяжении года или даже дольше. Вследствие этого всё меньше игр являются прибыльными и всё меньше разработчиков способны выпустить собственную игру, что в свою очередь, приводит к меньшей готовности рисковать и меньшему разнообразию на рынке игр. Процедурная генерация контента способна частично устранить эту проблему, ведь многие из дорогостоящих сотрудников, необходимых в этом процессе являются дизайнерами и художниками, а не программистами. Компания по разработке игр, которая смогла бы заменить некоторых художников и дизайнеров алгоритмами, получила бы конкурентное преимущество, поскольку была бы способна производить игры быстрее и дешевле, сохраняя при этом качество.

Более того, методы процедурной генерации способны значительно повысить среднее игровое время. Ведь если игра реализует алгоритм, способный генерировать игровой контент со скоростью, с которой он потребляется, то в принципе нет причин, почему она должна завершаться. У игроков не будут заканчиваться уровни для прохождения, области для исследования, оружие для использования, персонажи для взаимодействия.

Ещё более захватывающей является идея игрового контента, который бы соответствовал вкусам игрока. Используя нейронные сети и методы их эволюции, можно создать систему процедурной генерации контента, которая бы подстраивалась под предпочтения игрока. 

Очевидно, чтобы генерировать что-то, нужно знать что именно требуется получить. Какие свойства артефакта являются желаемыми? Какие будут нарушать игровой процесс и их следует исключить? Какой алгоритм использовать?

Данная работа посвящена исследованию методов нейроэволюции и их применению для процедурной генерации контента на примере генерации игрового оружия.

\pagebreak

\specialsection{Постановка задачи}

Целью данной работы является разработка программного инструмента, интегрированного в игровой движок Unity и предназначенного для процедурной генерации игрового оружия с уникальными траекториями снарядов.

Разрабатываемая система призвана облегчить разработку 2D игр в жанре шутеры и должна удовлетворять следующим требованиям:
\begin{enumerate}
    \item Система должна иметь достаточное количество настраиваемых параметров оружия, чтобы пользователь мог генерировать именно то, что ему нужно.
    \item Система не должна требовать больших усилий со стороны пользователя для генерации оружия.
    \item Система должна иметь качественный пользовательский интерфейс, работающий внутри Unity.
    \item Система должна иметь демонстрационные игровые сцены, показывающие возможные сценарии использования.
\end{enumerate}

\vspace{5mm}

Целью также является создание прототипа мобильной 2D игры, которая бы использовала эту систему. Требования для игры:
\begin{enumerate}
    \item Интерфейс, позволяющий игроку принимать или отказываться от оружия, а также редактировать некоторые его параметры.
    \item Два джойстика: один для движения, другой для стрельбы.
    \item Наличие стреляющего врага, при этом вражеские снаряды и снаряды игрока должны иметь возможность сталкиваться.
\end{enumerate}

\pagebreak

\specialsection{Обзор литературы}

В исследованиях\cite{s4,s5} проводится обзор различных методов нейроэволюции, производится их классификация, вводятся и объясняются важные понятия, такие как: прямая и косвенная схема кодирования нейронных сетей, пространство поиска, виды нейронных сетей и прочие. В работах\cite{s1,s16} в подробностях освещаются конкретные подходы: NEAT и HyperNEAT. Даются рекомендации по применению этих подходов, их недостатки и преимущества.

В статьях\cite{s2,s3} описывается идея системы частиц, управляемых нейронной сетью, и процесс эволюции таких систем под воздействием выбора пользователя. Была представлена программа на C++ с реализацией этой системы. Эти наработки были задействованы при создании игры GAR. В статье\cite{s15}, посвященной этой игре, описывается алгоритм генерации игрового оружия, производится классификация генерируемого оружия, указываются возникшие проблемы и предлагаются возможные варианты их решения.

Работы\cite{s6,s11} дают краткий обзор методов процедурной генерации контента, производят их классификацию, определяют терминологию и предлагают рекомендации для успешного создания генератора.


