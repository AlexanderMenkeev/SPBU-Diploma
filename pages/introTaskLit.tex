%! suppress = LineBreak
\specialsection{Введение}

С тех пор как были изобретены компьютерные игры, количество человеко-месяцев, которые уходят на разработку успешной коммерческой игры, постоянно увеличивалось. Сейчас является нормой, что игра разрабатывается сотнями людей на протяжении года или даже дольше. Вследствие этого всё меньше игр являются прибыльными и всё меньше разработчиков способны выпустить собственную игру, что в свою очередь, приводит к меньшей готовности рисковать и меньшему разнообразию на рынке игр. Процедурная генерация контента способна частично устранить эту проблему, ведь многие из дорогостоящих сотрудников, необходимых в этом процессе являются дизайнерами и художниками, а не программистами. Компания по разработке игр, которая могла бы заменить некоторых художников и дизайнеров алгоритмами, имела бы конкурентное преимущество, поскольку игры могли бы производиться быстрее и дешевле при сохранении качества.

Более того, методы процедурной генерации способны значительно повысить среднее игровое время. Ведь если игра реализует алгоритм, способный генерировать игровой контент со скоростью, с которой он <<потребляется>>, то в принципе нет причин, почему игра должна завершаться. У игроков не будут заканчиваться уровни для прохождения, оружие для использования, персонажи для взаимодействия, области для исследования.

Ещё более захватывающей является идея игрового контента, который бы соответствовал вкусам конкретного игрока. Комбинируя нейронные сети и методы их эволюции, можно создать систему процедурной генерации контента, которая бы подстраивалась под предпочтения игрока.

\pagebreak

\specialsection{Постановка задачи}

Целью данной работы является разработка программного инструмента, интегрированного в игровой движок Unity и предназначенного для процедурной генерации игрового оружия с уникальными траекториями снарядов.

Разрабатываемая система призвана облегчить разработку 2D игр в жанре шутеры и должна удовлетворять следующим требованиям:
\begin{enumerate}
    \item Система должна иметь достаточное количество настраиваемых параметров оружия, чтобы пользователь мог генерировать именно то, что ему нужно.
    \item Система не должна требовать больших усилий со стороны пользователя для генерации оружия.
    \item Система должна иметь качественный пользовательский интерфейс, работающий внутри Unity.
    \item Система должна иметь демонстрационные игровые сцены, показывающие возможные сценарии использования.
\end{enumerate}

Кроме того, целью является создание прототипа мобильной игры, которая бы использовала эту систему.

\pagebreak

\specialsection{Обзор литературы}

В рамках спецификации современных стандартов, базовые сценарии поведения пользователей призваны к ответу. Банальные, но неопровержимые выводы, а также представители современных социальных резервов формируют глобальную экономическую сеть и при этом - представлены в исключительно положительном свете.

Есть над чем задуматься: предприниматели в сети интернет будут описаны максимально подробно. Приятно, граждане, наблюдать, как сторонники тоталитаризма в науке заблокированы в рамках своих собственных рациональных ограничений. Есть над чем задуматься: некоторые особенности внутренней политики объявлены нарушающими общечеловеческие нормы этики и морали. Как принято считать, тщательные исследования конкурентов смешаны с неуникальными данными до степени совершенной неузнаваемости, из-за чего возрастает их статус бесполезности.

Лишь предприниматели в сети интернет, которые представляют собой яркий пример континентально-европейского типа политической культуры, будут преданы социально-демократической анафеме. Есть над чем задуматься: стремящиеся вытеснить традиционное производство, нанотехнологии являются только методом политического участия и ограничены исключительно образом мышления! Разнообразный и богатый опыт говорит нам, что постоянный количественный рост и сфера нашей активности напрямую зависит от новых предложений.
