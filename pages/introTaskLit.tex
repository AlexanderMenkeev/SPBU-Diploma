\specialsection{Введение}

Для того чтобы вдохнуть жизнь в виртуальные игровые миры, нужно
создать уровни, модели, текстуры и прочий игровой контент, что требует слаженных усилий художников и разработчиков. Это особенно важно для массовых многопользовательских ролевых онлайн игр (MMORPG), в которых необходим постоянный приток нового контента, чтобы удерживать интерес игроков.
Для решения данной проблемы существуют два основных подхода:
1. Предоставлять игрокам инструменты, с помощью которых они смогут создавать свой контент и делиться им с остальными;
2. Генерировать контент случайным образом. У этих подходов есть недостатки. Для того чтобы игроки могли создавать контент самостоятельно с помощью инструментов, они должны обладать высоким уровнем квалификации и специализированным знанием дизайна. Случайная генерация плоха тем, что она не учитывает предпочтений игроков и должна быть сильно ограничена, чтобы исключить совсем не интересный контент. По этой причине требуются дальнейшие исследования в этой области.

\pagebreak

\specialsection{Постановка задачи}

Целью научно-исследовательской практики является разработка программного инструмента, интегрированного в игровой движок Unity и предназначенного для процедурной генерации игрового оружия с уникальными траекториями снарядов.

Разрабатываемая система призвана облегчить разработку 2D игр в жанре шутеры и должна удовлетворять следующим требованиям:
\begin{enumerate}
    \item Система должна иметь достаточное количество настраиваемых параметров оружия, чтобы пользователь мог генерировать именно то, что ему нужно.
    \item Система не должна требовать больших усилий со стороны пользователя для генерации оружия.
    \item Система должна иметь качественный пользовательский интерфейс, работающий внутри Unity.
    \item Система должна иметь демострационные игровые сцены, показывающие возможные сценарии использования.
\end{enumerate}

Кроме того, целью практики является дальнейшая публикация разработанного продукта в Unity Asset Store. В связи с чем возникают следующие задачи:
\begin{enumerate}
    \item Необходимо написать документацию программного продукта на английском языке.
    \item Необходимо подготовить демонстрационные материалы. К примеру, WebGL сборка проекта, которую можно запускать в браузере, скриншоты, видео.
\end{enumerate}

%Целью научно-исследовательской работы является создание 2D игры,
%в которой будет реализована система процедурной генерации контента, которая учитывает предпочтения игрока. Для этой цели будут использованы искусственные нейронные сети. Задачи работы:
%1. Изучить виды нейронных сетей и методы нейроэволюции. 2. Разработать теоретическое решение проблемы процедурной генерации контента для конкретной игры.
%3. Использовать полученное теоретическое решение при создании игры.

\pagebreak

\specialsection{Обзор литературы}

В рамках спецификации современных стандартов, базовые сценарии поведения пользователей призваны к ответу. Банальные, но неопровержимые выводы, а также представители современных социальных резервов формируют глобальную экономическую сеть и при этом - представлены в исключительно положительном свете.

Есть над чем задуматься: предприниматели в сети интернет будут описаны максимально подробно. Приятно, граждане, наблюдать, как сторонники тоталитаризма в науке заблокированы в рамках своих собственных рациональных ограничений. Есть над чем задуматься: некоторые особенности внутренней политики объявлены нарушающими общечеловеческие нормы этики и морали. Как принято считать, тщательные исследования конкурентов смешаны с неуникальными данными до степени совершенной неузнаваемости, из-за чего возрастает их статус бесполезности.

Лишь предприниматели в сети интернет, которые представляют собой яркий пример континентально-европейского типа политической культуры, будут преданы социально-демократической анафеме. Есть над чем задуматься: стремящиеся вытеснить традиционное производство, нанотехнологии являются только методом политического участия и ограничены исключительно образом мышления! Разнообразный и богатый опыт говорит нам, что постоянный количественный рост и сфера нашей активности напрямую зависит от новых предложений.
