%! suppress = LineBreak
%! suppress = MissingLabel
\section{Методы эволюции нейронных сетей}\label{sec:evolutionMethods}

Существует множество алгоритмов для эволюции нейронных сетей. Они классифицируются по схемам кодирования и тому, модифицируют ли они веса или топологию или и то, и другое. Однако не все они пользуются популярностью и имеют хорошие библиотеки с реализацией. Можно выделить два алгоритма NEAT\cite{s1} и HyperNEAT\cite{s1}, которые находят своё применение до сих пор и имеют регулярно обновляемые библиотеки.

\subsection{NEAT}

Название этого алгоритма расшифровывается как нейроэволюция расширяющихся топологий (NeuroEvolution of Augmenting Topologies). Это метод, предназначенный для эволюции искусственных нейронных сетей с помощью генетического алгоритма. Главная идея NEAT заключается в том, что эволюцию наиболее эффективно начинать с маленьких, простых сетей, которые постепенно становятся всё более сложными с каждым поколением.

Алгоритм основан на трех ключевых принципах. Во-первых, для того, чтобы позволить структурам нейронных сетей усложняться с течением поколений, необходим метод отслеживания того, какой ген является каким. В противном случае в последующих поколениях будет неясно, какая особь с какой совместима и как их гены должны быть объединены для получения потомства. NEAT решает эту проблему, присваивая уникальную историческую метку каждому новому элементу структуры сети, который появляется в результате структурной мутации. Историческая метка – это число, присвоенное каждому гену в соответствии с порядком его появления в ходе эволюции. Эти числа наследуются без изменений во время скрещиваний и позволяют NEAT выполнять скрещивание без необходимости трудоемкого топологического анализа. Таким образом геномы различной организации и размеров остаются совместимыми на протяжении всей эволюции, что решает ранее открытую проблему сопоставления различных топологий в эволюционирующей популяции. 

Во-вторых, NEAT разделяет популяцию на виды таким образом, что отдельные особи конкурируют в основном внутри своих собственных ниш, а не с популяцией в целом. Благодаря этому топологические инновации защищены и имеют время оптимизировать свою структуру, прежде чем конкурировать с другими нишами в популяции. NEAT использует исторические метки на генах, чтобы определить, к какому виду принадлежат различные особи. 

В-третьих, в отличие от других существующих систем эволюции нейронный сетей алгоритм NEAT начинается с однородной популяции простых сетей без скрытых узлов. Возникает новая топологическая структура по мере того, как происходят структурные мутации, и выживают только те структуры, которые признаны полезными в результате оценки функцией приспособленности. Таким образом, NEAT выполняет поиск топологических структур, начиная с самых простых, и находит оптимальную структуру для решения задачи.



Плюсы алгоритма:
\begin{enumerate}[--]
    \item Наличие библиотеки на C\#, которая регулярно обновляется.
    \item Подходит для трудно формализируемых задач, где функция потерь и функция приспособленности могут быть не определены.
\end{enumerate}

Минусы алгоритма:
\begin{enumerate}[--]
    \item Медленно сводится к оптимальному решению, если условия задачи очень сложны. 
    \item 
\end{enumerate}

\subsection{HyperNEAT}


Плюсы алгоритма:
\begin{enumerate}[--]
    \item Наличие библиотеки на C++, которая регулярно обновляется.
    \item Подходит для трудно формализируемых задач, где функция потерь и функция приспособленности могут быть не определены.
\end{enumerate}

Минусы алгоритма:
\begin{enumerate}[--]
    \item Намного медленнее, чем NEAT, поскольку требуется больше шагов для создания нейронных сетей на каждом из поколений.
    \item Изменяет только веса, топология нейронной сети определяется пользователем с помощью <<субстрата>>.
\end{enumerate}








