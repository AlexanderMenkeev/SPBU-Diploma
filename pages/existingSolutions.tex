%! suppress = LineBreak
\section{Обзор существующих решений}\label{sec:existingSolutions}

Есть множество игр, в которых реализована процедурная генерация игрового оружия. Из них можно выделить две:

\begin{enumerate}
    \item Borderlands\cite{s14}\\ Это 3D шутер от первого лица с огромным разнообразием оружия. Система процедурной генерации оружия в этой игре основана на том, что каждое оружие представляется 35 деталями, среди них: ствол, глушитель, рукоять, цевье, конденсатор и так далее. Каждая деталь меняет некоторые характеристики оружия, такие как: темп стрельбы, урон, наличие силового щита и т.п., при этом уникальных деталей в инвентаре игрока может быть до 1500. Кроме того, есть классификация оружия по его типу, редкости и производителю, что ещё больше увеличивает количество возможных вариантов.
    
    Плюсы системы из этой игры:
    \begin{enumerate}[--]
        \item Крайне большое количество генерируемого оружия.
        \item Уникальный внешний вид оружия за счет комбинации различных деталей.
    \end{enumerate}
    
    Минусы системы из этой игры:
    \begin{enumerate}[--]
        \item В системе не предусмотрено получение обратной связи от игрока, поэтому он не способен никак повлиять на генерацию оружия.
        \item Генерируемое оружие не сильно различается траекториями полета пуль.
    \end{enumerate}
\end{enumerate}


\begin{enumerate}
    \setcounter{enumi}{1}
    \item Galactic Arms Race (GAR)\cite{s15}\\ Это 2.5D космический шутер с видом сверху (topdown). Система процедурной генерации оружия в этой игре основана на том, что каждое оружие представляется нейронной сетью с помощью системы частиц. Затем эти нейронные сети подвергаются эволюции и тем самым создается новое оружие.
    
    Плюсы системы из этой игры:
    \begin{enumerate}[--]
        \item Система подстраивается под предпочтения игроков.
        \item Уникальные траектории снарядов.
    \end{enumerate}

    Минусы системы из этой игры:
    \begin{enumerate}[--]
        \item Для нормальной работы необходима группа игроков, в одиночном режиме оружие эволюционирует медленно.
        \item Генератор иногда выдает неприемлемое оружие (Снаряды либо не движутся, либо колеблются)
    \end{enumerate}
\end{enumerate}

\textbf{Замечание.} Приведенные выше примеры являются коммерческими играми и поэтому получить исходный код их алгоритмов генерации оружия не получится. Система же которая будет разработана в ходе этой работы будет в последствии выложена в открытый доступ. Тем самым каждый желающий разработчик Unity сможет интегрировать её в свой проект и использовать для своей игры. Если говорить именно про систему, а не про игру, то её аналогов в Unity Asset Store найдено не было.




