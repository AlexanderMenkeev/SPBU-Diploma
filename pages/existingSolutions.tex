%! suppress = LineBreak
%! suppress = MissingLabel
\section{Обзор существующих решений}\label{sec:existingSolutions}

Есть множество игр, в которых реализована процедурная генерация игрового оружия. Из них можно выделить две с наиболее интересными системами: Borderlands\cite{s14} и GAR\cite{s15}.

\vspace{1mm}

\textbf{Borderlands} -- это 3D шутер от первого лица с огромным разнообразием оружия. Система процедурной генерации оружия в этой игре основана на том, что каждое оружие представляется 35 деталями, среди них: ствол, глушитель, рукоять, цевье, конденсатор и другие. Каждая деталь меняет некоторые характеристики оружия: темп стрельбы, урон, наличие силового щита и т.п. При этом уникальных деталей в инвентаре игрока может быть до 1500. Также есть классификация оружия по его типу, редкости и производителю, что ещё больше увеличивает количество возможных вариантов. 
%
\\Плюсы системы из этой игры:
\begin{enumerate}[--]
    \item Крайне большое количество генерируемого оружия.
    \item Уникальный внешний вид оружия за счет комбинации различных деталей.
\end{enumerate} 
%
Минусы системы из этой игры:
\begin{enumerate}[--]
    \item В системе не предусмотрено получение обратной связи от игрока, поэтому он не способен никак повлиять на генерацию оружия.
    \item Генерируемое оружие не сильно различается траекториями полета пуль.
\end{enumerate}

\vspace{1mm}

\textbf{Galactic Arms Race} -- это 2.5D космический шутер с видом сверху. Система процедурной генерации оружия в этой игре основана на том, что каждое оружие представляется нейронной сетью с помощью системы частиц. Затем эти нейронные сети подвергаются эволюции и тем самым создается новое оружие.
%
\\Плюсы системы из этой игры:
\begin{enumerate}[--]
    \item Система подстраивается под предпочтения игроков.
    \item Уникальные траектории снарядов.
\end{enumerate}
%
Минусы системы из этой игры:
\begin{enumerate}[--]
    \item Для нормальной работы необходима группа игроков, так как в одиночном режиме оружие эволюционирует медленно.
    \item Игрок не имеет возможности редактировать параметры оружия.
    \item Генератор иногда выдает неприемлемое оружие, снаряды которого либо не движутся, либо колеблются так быстро, что коллизии с врагами не фиксируются игрой.
\end{enumerate}

\vspace{1mm}

\textbf{Вывод.} В результате исследования рынка игр было выявлено, что большинство игр с процедурной генерацией оружия, используют системы, похожие на систему из Borderlands, то есть оружие собирается из деталей и тем самым происходит модификация его характеристик. В некоторых играх встречалась простая рандомизация параметров оружия, где диапазон случайных значений зависел от его типа или редкости. При этом GAR оказалась единственной игрой, где у оружия менялась сама траектория полёта снарядов.

Стоит отметить, что приведенные выше примеры являются коммерческими играми и поэтому получить исходный код их алгоритмов генерации оружия не получится. В Unity Asset Store не было найдено аналогов разрабатываемой в ходе этой работы системы. Данная система в конечном итоге будет выложена в открытый доступ, тем самым каждый разработчик Unity сможет интегрировать её в свой проект и использовать для разработки своей игры.




